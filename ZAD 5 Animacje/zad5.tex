\documentclass[10pt,a4paper]{article}
\usepackage[utf8]{inputenc}
\usepackage{amsmath}
\usepackage{polski}
\usepackage{amsfonts}
\usepackage[T1]{fontenc}
\usepackage{amssymb}
\usepackage{graphicx}
\author{Krystian Barczak}
\title{Macierze}
\frenchspacing
\begin{document}
\maketitle
\tableofcontents
\newpage
\section{Macierze 1}
Mnożenie macierzy jest możliwe jedynie w przypadku, gdy dana macierz A ma tyle samo wierszy co macierz B kolumn.
\begin{equation}

	\begin{bmatrix}
 		1&  2& 3\\ 
 		4&  5& 6\\ 
 		7&  8& 9
	\end{bmatrix}
	\cdot 
	\begin{bmatrix}
	 	1&  2& 3\\ 
		4&  5& 6\\ 
		7&  8& 9
	\end{bmatrix}
	= 
	\begin{bmatrix}
	 	30&  36& 42\\ 
		66&  81& 96\\ 
		102&  126& 150
	\end{bmatrix}
	\cite{pa}
\end{equation}


\newpage
\section[Kolejny rozdział]{Macierze 2}
Treść drugie rozdziału.


\begin{thebibliography}{99}
\bibitem{pa}Macierz 1
\end{thebibliography}
\end{document}