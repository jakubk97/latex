\documentclass{beamer}
\usepackage[T1]{fontenc}
\usepackage[polish]{babel}
\usepackage[utf8]{inputenc}
\usetheme{Antibes}
\usepackage{array}
\usepackage{multirow}
\usepackage{listings}
\usepackage{longtable}



\title{PAKIET ARRAY}
\subtitle{tworzenie tabel w LaTeX'ie}
\author{Adam Czerwiński i Jakub Karmański}
\institute{POLSL}
\date{28.10.2019}
 
\renewcommand*\contentsname{Summary}
 
\begin{document}
 
\maketitle
 
\tableofcontents
 
\section{Wprowadzenie}

\begin{frame}[fragile]
Array to pakiet do LaTeX’a który udostępnia narzędzia do produkowania rozbudowanych tabel. 
Same tabele nie potrzebują pakietu 'array' do stworzenia tablicy, lecz w przypadku chęci ingerowania w np. długość czy kolor komórek pakiet ten jest wymagany.

W celu jego dodania do naszego pliku używamy:

\begin{lstlisting}
\usepackage{array}
\end{lstlisting}
\end{frame}

\begin{frame}[fragile]

Aby zdefiniować tablice:
\begin{lstlisting}
\begin{tabular}
\end{tabular} 
\end{lstlisting}
To oświadcza, że w tabeli zostaną użyte trzy kolumny oddzielone pionową linią. Każde c oznacza, że zawartość kolumny zostanie wyśrodkowana, możesz także użyć r, aby wyrównać tekst do prawej i l, aby wyrównać do lewej:
\begin{lstlisting}
{ |c|c|c| }
\end{lstlisting}
Spowoduje to wstawienie poziomej linii na górze stołu i na dole. Nie ma ograniczeń co do liczby przypadków użycia \ hline :
\begin{lstlisting}
\hline
\end{lstlisting}
Każdy 'Et' jest separatorem komórek, a podwójny ukośnik \\ ustawia koniec tego wiersza.
\begin{lstlisting}
cell1 & cell2 & cell3 \\
\end{lstlisting}
\end{frame}

\section{Tworzenie prostej tabeli w LaTeX}
\begin{frame}[fragile]
 
Poniżej możemy zobaczyć najprostszy działający przykład tabeli bez użycia pakietu array

\begin{lstlisting}
\begin{tabular}{ c c c }
 cell1 & cell2 & cell3 \\ 
 cell4 & cell5 & cell6 \\  
 cell7 & cell8 & cell9    
\end{tabular}
\end{lstlisting}

\begin{center}
\begin{tabular}{ c c c }
 cell1 & cell2 & cell3 \\ 
 cell4 & cell5 & cell6 \\  
 cell7 & cell8 & cell9    
\end{tabular}
\end{center} 

Gdzie { c c c } - jest definicją kolumn naszej tablicy.Litera 'c' pochodzi od słowa 'center' i jest ona wymagana podczas definiowania kolumn naszej tablicy
 
\end{frame}

\begin{frame}[fragile]

Obramowanie:
\begin{lstlisting}
\begin{tabular}{ |c|c|c| } 
 \hline
 cell1 & cell2 & cell3 \\ 
 cell4 & cell5 & cell6 \\ 
 cell7 & cell8 & cell9 \\ 
\end{tabular}
\end{center}
\end{lstlisting}

\begin{center}
\begin{tabular}{ |c|c|c| } 
 \hline
 cell1 & cell2 & cell3 \\ 
 cell4 & cell5 & cell6 \\ 
 cell7 & cell8 & cell9 \\ 
\end{tabular}
\end{center}
\end{frame}

\section{Tablice o stałej długości}

\begin{frame}[fragile]
\begin{lstlisting}
\begin{tabular}{ | m{1cm} | m{1cm}| m{1cm} | } 
\hline
cell1 dummy text dummy text dummy text& cell2 & cell3 \\ 
\hline
cell1 dummy text dummy text dummy text & cell5 & cell6 \\ 
\hline
cell7 & cell8 & cell9 \\ 
\hline
\end{tabular}
\end{lstlisting}
\end{frame}

\begin{frame}
 \begin{center}
\begin{tabular}{ | m{2cm} | m{1cm}| m{3cm} | } 
\hline
cell1 & cell2 & cell3 \\ 
\hline
cell1 & cell5 & cell6 \\ 
\hline
cell7 & cell8 & cell9 \\ 
\hline
\end{tabular}
\end{center}
\end{frame}

\section{Lączenie kolumn i wierszy}



\begin{frame}
\begin{center}
\begin{tabular}{ |p{2cm}|p{2cm}|p{2cm}|  }
 \hline
 \multicolumn{3}{|c|}{Połączenie 3 kolumn} \\
 \hline
 Kol1&Kol2&Kol3\\
 \hline
 1 & 2 & 3\\
 1 & 2 & 3\\
 1 & 2 & 3\\
 1 & 2 & 3\\
 \hline
\end{tabular}
\end{center}
\end{frame}

\begin{frame}

 multicolumn  {3}  {|c|}  {Połączenie 3 kolumn}
 
 usepackage	{multirow}
\end{frame}

\section{Tabele wielostronicowe}
\begin{frame}[fragile]
 
Zachowanie longtable jest podobne do domyślnego tabelarycznego, ale generuje tabele, które można podzielić za pomocą standardowego algorytmu podziału strony LATEX. Istnieją cztery elementy specyficzne dla longtable.
\end{frame}

\begin{frame}

1.endfirsthead\\
Wszystko powyżej tego polecenia pojawi się na początku tabeli, na pierwszej stronie.\\

2.endhead\\
Cokolwiek umieścisz przed tym poleceniem i poniżej endfirsthead, wyświetli się u góry tabeli na każdej stronie oprócz pierwszej.\\

3.endfoot\\
Podobnie jak endhead, to co umieścisz po endhead i przed tym poleceniem pojawi się na dole tabeli na każdej stronie oprócz ostatniej.\\

4.endlastfoot\\
Podobne do endfisthead. Elementy po endfoot i przed tym poleceniem będą wyświetlane na dole tabeli, ale tylko na ostatniej stronie, na której pojawia się tabela.\\
Przykład w pliku...
\end{frame}

\section{Pozycjonowanie tablic}
\begin{frame}
 
\end{frame}

\section{Podpisy,etykiety i odniesienia}
\begin{frame}
 
\end{frame}

\section{Lista tabel}
\begin{frame}
 
\end{frame}

\section{Zmiana wyglądu tabeli}
\begin{frame}
 
\end{frame}

\section{Linki}
\begin{frame}
 www.overleaf.com/learn/latex/Tables
\end{frame}

\section{Zadanie}
\begin{frame}
 Przygotować tabele,która będzie miała w sobie połączone kolumny oraz wiersze.
 
\end{frame}
 
 
\end{document}