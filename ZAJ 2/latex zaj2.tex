\documentclass{beamer}
\setbeamercovered{transparent}
\usepackage[utf8]{inputenc}
\usetheme{Antibes} 

\usepackage{array}
\usepackage{multirow}

\usepackage{xcolor,colortbl}
\definecolor{green}{rgb}{0.1,0.1,0.1}
\newcommand{\done}{\cellcolor{teal}done}  %{0.9}
\newcommand{\hcyan}[1]{{\color{teal} #1}}

\newcommand{\colorek}{\cellcolor{teal}colorek}

%Information to be included in the title page:
\title{Nauka beamera}
\author{Jakub Karmański}
\institute{POLSL}
\date{14.10.2019}
 
 
 
\begin{document}
 
\frame{\titlepage}
 
\begin{frame}
\frametitle{Strona 1}
 \framesubtitle{podtytuł}
\begin{itemize}
\color{blue}
\item<1> \textbf{a}
\item<2> \textit{b}
\item<4> \textcolor{red}{\textbf{d}}
\item<3> \underline{c}
\end{itemize}
 
\end{frame}

\begin{frame}
\frametitle{Strona 2}
 \framesubtitle{podtytuł}
\begin{itemize}
 \item<1-> \href{https://pl.wikipedia.org/wiki/LaTeX}{\beamergotobutton{Link do wikipedii o latexie}}
 \item<2-> Dowolny tekst
\end{itemize}
 \end{frame}
 

\begin{frame}
\frametitle{Strona 3}
 \framesubtitle{podtytuł}

	\begin{center}
\begin{tabular}{ | m{1cm} | m{1cm}| m{1cm} | } 
\hline
\onslide<1>{a & b & c} \\ 
\hline
\onslide<2>{d & e & f} \\ 
\hline
\onslide<2>{g & h & i} \\
\hline
\end{tabular}
\end{center}

 \end{frame} 
 
 
\begin{frame}
\frametitle{Strona 4}
\framesubtitle{podtytuł}
 
\begin{center}
\begin{table}[ht]
\caption{Multi-column and multi-row table}
\begin{center}
\begin{tabular}{ccc}
    \hline
    \multicolumn{2}{c}{\multirow{2}{*}{Multi-col-row}}&X\\
    \multicolumn{2}{c}{}&X\\
    \hline
    X&X&X\\
    \hline
\end{tabular}
\end{center}
\label{tab:multicol}
\end{table}
\end{center}
\end{frame}


\begin{frame}
\frametitle{Strona 5}
\framesubtitle{podtytuł}
 
\begin{center}
\begin{tabular}{ll}
\done & other \\
\hline
brak & \colorek \\
\end{tabular}
\end{center}
\end{frame}
 
 
 \end{document}