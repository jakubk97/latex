\documentclass[table]{beamer}
\usepackage[T1]{fontenc}
\usepackage[polish]{babel}
\usepackage[utf8]{inputenc}
\usetheme{Antibes}

\usepackage{array}

\usepackage{multirow}
\usepackage{listings}
\usepackage{longtable}
\usepackage{tabularx}
\usepackage{xcolor}
\usepackage{colortbl}

\setlength{\arrayrulewidth}{0.7mm}
\setlength{\tabcolsep}{15pt}
\renewcommand{\arraystretch}{1.3}



\title{PAKIET ARRAY}
\subtitle{tworzenie tabel w LaTeX'ie}
\author{Jakub Karmański}
\institute{POLSL}
\date{28.10.2019}
 

\begin{document}
 
\maketitle

\listoftables %nie pojawia się ponieważ w bemerze lista tablic nie działa

\begin{frame}
\begin{tabularx}{0.8\textwidth} { 
  | >{\raggedright\arraybackslash}X 
  | >{\centering\arraybackslash}X 
  | >{\raggedleft\arraybackslash}X | }
 \hline
 item 11 & item 12 & item 13 \\
 \hline
 item 21  & item 22  & item 23  \\
\hline
\end{tabularx}
\end{frame}


\begin{frame}
\begin{center}
\begin{tabular}{ |c|c|c|c| } 
\hline
 \multicolumn{3}{|c|}{polaczone kolumny} \\
 \hline
col1 & col2 & col3 \\
\hline
\multirow{3}{4em}{polaczone wiersze} & cell2 & cell3 \\ 
& cell5 & cell6 \\ 
& cell8 & cell9 \\ 
\hline
\end{tabular}
\end{center}
\end{frame}



\begin{frame}
Odwołanie do tablicy:  \ref{table:1} (kliknij)
\end{frame}

\begin{frame}
\begin{table}[ht!]
\centering
\begin{tabular}{||c c c c||} 
 \hline
 Col1 & Col2 & Col2 & Col3 \\ [0.5ex] 
 \hline\hline
 1 & 6 & 87837 & 787 \\ 
 2 & 7 & 78 & 5415 \\
 3 & 545 & 778 & 7507 \\
 4 & 545 & 18744 & 7560 \\
 5 & 88 & 788 & 6344 \\ [1ex] 
 \hline
\end{tabular}
\caption{Podpis tablicy}
\label{table:1}
\end{table}
\end{frame}

\begin{frame}
\begin{tabular}{ |p{2cm}|p{2cm}|p{2cm}|  }
\hline
1 & 2 & 3 \\
\hline
4 & 5 & 6  \\
7 & 8 & 9 \\
\hline
\end{tabular}
\end{frame}

\begin{frame}
\newcolumntype{s}{>{\columncolor[HTML]{AAACED}} p{3cm}}
 
\arrayrulecolor[HTML]{DB5800}
\begin{tabular}{ |s|p{2cm}|p{2cm}|  }
\hline
\rowcolor{lightgray} \multicolumn{3}{|c|}{Polaczone kolumny} \\
\hline
1 & 2 & 3 \\
\hline
4 & 5 & \cellcolor[HTML]{AA0044}6 \\
\rowcolor{pink}
7 & 8 & 9 \\
\hline
\end{tabular}
\end{frame}

\begin{frame}
\newcolumntype{s}{>{\columncolor[HTML]{AAACED}} p{3cm}}
 
 \arrayrulecolor{black}
  \begin{tabular}{ |l!{\color{green}\vrule}l|}
    \arrayrulecolor{red}\hline
    1 & 2\\\arrayrulecolor{blue}\hline
    3 & 4\\\arrayrulecolor{yellow}\hline
  \end{tabular}
\end{frame}

\begin{frame}
\begin{table}[H]
 \centering
 \begin{tabular}{|l!{\color{red}\vrule}l|l|}
 \hline
 & \multicolumn{1}{c|}{Text}  \\
 \hline
 \parbox[t]{2mm}{\multirow{2}{*}{\rotatebox[origin=c]{90}{text}}} & text\\
 & text \\\arrayrulecolor{blue}\hline
 & text \\\arrayrulecolor{black}
 \hline
 \end{tabular}
 \end{table}
\end{frame}

\end{document}